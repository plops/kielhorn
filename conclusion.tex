\chapter{Conclusion}
In this work I document the development of a spatio-angular
microscope. An important part of the project was to build a working
prototype, which I did in several steps.

Initially, I used a display for spatial control that was connected to
the graphics card. This was sufficient for some proof-of-concept
measurements but there were some problems with triggering. However, I
eventually replaced it with a display having a controller with local
storage. This enables the capturing of image stacks at a sufficient
speed (camera running at \unit[30]{Hz}) with precise triggering and
the light is only sent into the specimen when a camera exposure is
integrated.

However, the preparation time for each stack is quite high.  Uploading
images to the two SLMs is quite slow and somewhat limits the range of
experiments that can be done.

Nevertheless, our microscope prototype provides a different way of
illumination control compared to other approaches in the
literature. In particular, our approach can not control simultaneous
angles in the sample, as seen in the micro-lens based light field
approach \citep{Levoy2009}. That, however, gives our system more
flexibility. While the trade-off between the angular and spatial
resolutions is fixed in the light-field approach, the resolutions can
be easily adapted in our microscope.

The prototype should now be used to investigate the phototoxic effects
biological specimen.

The microscope could be improved in several ways.  Most effort should
go into better illumination strategies. Up to now I alwas illuminated
single target beads. It would be useful to find clusters of targets,
that can be illuminated by the same angle.


