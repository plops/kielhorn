\chapter{Conclusion}
In this work we document the development of a spatio-angular
microscope. An important part was to build a working prototype. We did
this in several steps.


Initially we used a display for spatial control that was connected to
the graphics card. This was sufficient for some proof of concept
measurements but there were some problems with triggering.  But
eventually we replaced it by a display with a controller that contains
local storage. This enables capturing of image stacks at sufficient
speed with precise triggering and light is only sent into the
specimen, when a camera exposure is integrated.

However, the preparation time for each stack is quite high.  Uploading
images to the two SLMs is quite slow and somewhat limits the range of
experiments that can be done.

Still our microscope prototype provides a different way of
illumination control compared to other approaches in the
literature. Particularly our approach can't control simultaneous
angles in the sample like the micro-lens based light field approach
\citep{Levoy2009}. That, however, gives our system more
flexibility. Where their approach fixes the trade-off between the
angular and spatial resolutions, the resolutions can be easily adapted
in our microscope.
