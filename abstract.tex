\section*{Abstract}
\begin{summary}
  Photobleaching and phototoxicity pose a problem in live cell
  imaging. Fluorescence imaging induces reactive oxygen species in
  observed organisms which can alter the behaviour of the
  sample. Hence, minimising the light exposure is an important goal.

  We augment a wide field epifluorescence microscope with two spatial
  light modulators. By controlling the spatial excitation pattern and
  the angle of illumination, we can adapt the illumination to the
  specimen. In many cases, this technique will create exposures with
  reduced excitation of the out-of-focus fluorophores, resulting in
  better image quality and less phototoxicity.

  Our custom software is used to obtain an initial image stack of the
  specimen. Subsequent image sections are exposed with excitation
  patterns that account for the previous image stack. Depending
  upon the distribution of fluorophores, this adaptive exposure can
  considerably reduce photobleaching and phototoxicity.
\end{summary}
