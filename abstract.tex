\section*{Abstract}
\begin{summary}
Photobleaching and phototoxicity pose a problem in live cell
imaging. Fluorescence imaging induces reactive oxygen species in
observed organisms which can alter the behaviour of the sample, and so
minimising light exposure is an important goal.

We augment a widefield epifluorescence microscope with two spatial
light modulators. By controlling the spatial illumination pattern and
the angle at which illumination occurs, we achieve control of the
illumination pattern in three dimensions.

Our custom software is used to obtain an initial image stack of the
specimen. Subsequent image sections are exposed with excitation
patterns that take into account the previous image stack. Depending on
the distribution of fluorophores this adaptive exposure can
considerably reduce photobleaching and phototoxicity.
\end{summary}
