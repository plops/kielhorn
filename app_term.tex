\chapter{Construction of term symbols for molecular orbitals}
\label{sec:app_term}

The symmetric linear combination of the $1s$-orbitals of two atoms $A$
and $B$ is
\begin{align*}
  \sigma_g1s=\frac{1}{\sqrt2}(\sigma 1s_A+\sigma 1s_B).
\end{align*}
$\sigma_u$ is constructed as the difference of the two atomic
orbitals.  In general the symmetric molecular orbital $\sigma_g$ is
more stable as the electrons have a higher probability to be between
the nuclei.  The following quantum numbers describe the molecular wave
function:
\begin{description}
\item[$\Lambda\ ..$]
      
  Defined by $\mathbf{L}_z=\Lambda\hbar=|\sum\lambda_i|\hbar$ with
  projection $\lambda_i$ of the orbital angular momentum
  $\mathbf{l}_i$ of electron $i$ onto the nuclear axis. $\Lambda$ can
  take the values $0,1,2,\ldots$ and one writes the term symbols
  $\Sigma,\Pi,\Delta,\ldots$.
      
\item[$S\ ..$]
      
  Spin of all electrons $S=\sum\mathbf{m}_{si}$ in the molecular
  orbital.
      
\item[$\Omega\ ..$]
      
  Electronic angular momentum in direction of the nuclear axis.
      
\end{description}
These numbers are combined into the term symbol like this:
$\boxed{{}^{2S+1}\Lambda_\Omega}$.  Additionally one writes
e.g. $\Sigma^+$ if the molecular function is symmetric when mirrored
at a plane through the nuclei.  Furthermore one writes $\Sigma_g$ if
the sign of wave function stays the same when the molecule is inverted
at the point of symmetry.
    
Here are two examples of ``decoding'' the term symbols describing wave
functions of the hydrogen molecule:
    
\begin{description}
\item[$\sigma_g\sigma_g^1\Sigma_g^+$:]
      
  $\lambda_1=1,\ \lambda_2=-1,\ s_1=+,\ s_2=-$
      
  $\Lambda=|\lambda_1+\lambda_2|=|1-1|=0,\ S=s_1+s_2=0$
      
\item[$\sigma_g\pi_u^1\Pi_u$:]
      
  $\lambda_1=1,\ \lambda_2=-2,\ s_1=+,\ s_2=-$
      
  $\Lambda=|1-2|=1,\ S=0$
      
\end{description}
