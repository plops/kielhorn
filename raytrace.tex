% ~/from-hp2-notebook/0331/lens
% there is also code
\chapter{Raytracing for spatio-angular microscopy}
\label{sec:raytrace}
\renewcommand{\i}{\nvect i}

Here we give an overview of some useful equations for raytracing
through lens models. The design parameters of our microscope
objectives are not usually known to us. However, this is not an
unsurmountable problem as they can be represented using a simplified
model \citep{Hwang2008}. We use this to simulate the refraction at the
coverslip--medium interface for non-index matched media.

\section{Refraction at plane surface}
We begin by describing refraction at a plane surface\footnote{The
  equations are as in \citep{McClain1993}.}. The wavelength of the
light defines the length of the wave vector $\k_0$. The lengths of the
incident and transmitted wave vectors $\k_1$ and $\k_2$ are given by
the refractive index in their respective half space:
\begin{align}
  k_0&=2\pi/\lambda\\
  k_1&=n_1 k_0\\
  k_2&=n_2 k_0.
\end{align}
The normal $\n$ is directed in the opposite direction of the incident
wave vector $\k_1$. We define the transversal and normal component
vectors:
\begin{align}
  \k_{1n}&=(\k_1\n)\n\\ 
  \k_{1t}&=\k_1 - \k_{1n}.
\end{align}
Both of these components are perpendicular and during refraction the
transversal component of the wave vector is invariant:
\begin{align}
  k_2^2&=k_{2n}^2 + k_{2t}^2\\
  \k_{2t}&=\k_{1t}.
\end{align}
Using the two equations from above we can calculate the length of the
normal component of the transmitted wave vector $\k_2$:
\begin{align}
  k_2^2&=k_{2n}^2 + (\k_1 - \k_{1n})^2\\
  k_{2n}^2&=k_2^2-(\k_1-(\k_1\n)\n)^2\\
  &= k_2^2-(k_1^2-2(\k_1\n)^2+(\k_1\n)^2)\\
  &= k_2^2-k_1^2+(\k_1\n)^2.
\end{align}
Finally we can express the full transmitted wave vector $\k_2$ using
only known quantities:
\begin{align}
  \k_2&=\k_{1t}-\sqrt{k_2^2-k_1^2+(\k_1\n)^2}\n\\
  &=\k_1-(\k_1\n)\n-\sqrt{k_2^2-k_1^2+(\k_1\n)^2}\n.
\end{align}
We divide by $k_2$ with $\k_2/k_2=\t$ and $\k_1/k_2=\eta\,\i$ in order
to introduce unit direction vectors $\i$ and $\t$ for incident and
outgoing light. The relative index change across the interface is
$\eta=n_1/n_2$.
\begin{figure}
  \centering
  \input{refraction.eps_tex}
  \caption{Refraction at an interface transforms the incident wave
    vector $\k_1$ into the outgoing wave vector $\k_2$.}
\end{figure}
\begin{align}
  \t&=\eta\i-\eta(\i\n)\n-\sqrt{1-\eta^2+\eta^2(\i\n)^2}\n\\
  &=\boxed{\eta\i-\left(\eta\i\n+\sqrt{1-\eta^2(1-(\i\n)^2)}\right)\n}
\end{align}
When the radical in the square root is negative a reflection occurs
instead (TIRF). The tangential component is invariant and normal
component inverts the sign:
 \begin{align}
   \k_2&=\k_{1t}-\k_{1n}\\
   &=\k_1 - 2\k_{1n}\\
   &=\k_1-2(\k_1\n)\n\\
   \t&=\boxed{\i-2(\i\n)\n}
 \end{align}
\section{Intersection of a ray and a plane}
Let a ray start at a point $\s$ with direction $\hd$.  A plane
(defined by a point $\c$ and the normal $\n$) intersects this ray if
normal and ray direction are not perpendicular: $\n\,\hd\not=0$. The
distance between the plane and the origin is $h=\c\n$. We can define
the plane equation in Hesse normal form:
\begin{align}
  \r\n=h
\end{align}
We replace the coordinate $\r$ with the ray equation and solve for
the parameter $\tau$:
\begin{align}
  (\s+\tau\hd)\n&=h\\
  \s\n+\tau\hd\n&=h\\
  \tau&=\boxed{\frac{h-\s\n}{\hd\n}}
\end{align}
 \begin{figure}[!hbt]
   \centering
   \input{plane-intersection.eps_tex}
   \caption{Schematic for describing the plane-ray intersection.}
 \end{figure}
\section{Intersection of a ray and a sphere}
Let a ray start at a point $\s$ with direction $\hd$.  Let a sphere
sphere be centered in $\c$ with radius $R$. Their two equations
\begin{align}
  (\r-\c)^2&=R^2\\
  \r&=\s+\tau\hd
\end{align}
define the intersection points. Substitution of $\r$ results in a
quadratic equation for $\tau$:
\begin{align}
  (\s+\tau\hd-\c)^2&=R^2\\
  \l&:=\boxed{\s-\c}\\
  l^2+2\tau\l\hd+\tau^2-R^2&=0\\
  \tau^2+\underbrace{2\l\hd}_b\tau+\underbrace{l^2-R^2}_c&=0
\end{align}
\subsection{Solving the quadratic equation}
If the determinant $d$ is negative the ray misses the sphere and there
is no solution. If the determinant is zero the ray touches the
periphery and there is only one solution. A positive determinant
corresponds to two solutions. In order to prevent numerical errors the following solution should be used \citep{Press1997}: 
\begin{align}
  d&:=\boxed{b^2-4ac}\\
  q&:=\boxed{-\frac{b+\sqrt{d}\sign b}{2}}\\
  \tau&=\boxed{
  \begin{cases}
    \frac{q}{a} &\,\textrm{when}\,\abs{q}\approx 0\\ 
    \frac{c}{q} &\,\textrm{when}\,\abs{a}\approx 0\\
    (\frac{q}{a}, \frac{c}{q}) &\,\textrm{else}
  \end{cases}}
\end{align}
\section{Refraction on paraxial thin lens}
\begin{figure}[!hbt]
  \centering
  \input{lens-fwd.eps_tex}
  \caption{Construction of a ray on a thin lens. The incident beam
    with direction $\i$ hits the lens at the point $\vrho$.}
\end{figure}
The incident beam with direction $\i$ hits the lens at the point
$\vrho$. A line parallel to $\i$ through the center of the lens
defines the point on the focal plane, which will be intersected by the
transmitted ray $\r$ as well.

The triangle $ABC$ is similar to triangle $FOA$. All three angles are
identical because each of the lines are parallel:
$\overline{CB} \parallel \overline{OA} \parallel \vrho$,
$\overline{FA} \parallel \overline{CA}$ and $\overline{AB} \parallel
\overline{OF} \parallel \i$. The side $\overline{OF}$ is hypothenuse
of a right angled triangle. Its ancathete with respect to the angle
$\theta$ has length $f$. Therefor the we can deduce the length
$\abs{\overline{OF}}=f/\cos\theta$.

Between the two similar triangles, the following relation holds and
can be used to calculate the length $\abs{\overline{BC}}$:
\begin{align}
  \frac{\abs{\overline{BC}}}{\abs{\overline{BA}}}&=
  \frac{\abs{\overline{OA}}}{\abs{\overline{OF}}}\\
  \frac{\abs{\overline{CB}}}{1}&=
  \frac{\rho}{f/\cos(\theta)}.
\end{align}
Given its length, the vector $\overline{CB}$ can now calculated,
because we know its direction to be along $\vrho$. With this vector
and $\i$ we can now obtain the (arbitrarily scaled) transmitted vector
$\r'$. We could normalize it but it turns out to be useful for the
high NA immersion lens to find the vector $\r$, that ends in the focal
plane.  The procedure from above is condensed in the following
equations:
\begin{align}
  \vrho&=(x_0,y_0,0)^T=\rho (\cos\phi,\sin\phi,0)^T\\
  \phi&=\arctan(y_0/x_0)\\
  \cos\theta&=\boxed{\i\hz}\\
  \r'&=\i- \frac{\cos\theta}{f}\vrho\\
  \r&=\boxed{\frac{f}{\cos\theta} \i -\vrho}
\end{align}

\section{Refraction through oil objective (illumination)}
\begin{figure}[!hbt]
  \centering
  \input{obj-fwd.eps_tex}
  \caption{Construction of a ray on an high numerical aperture oil
    immersion objective. As opposed to a thin air lens the objective's
    focal length needs to be corrected by the focus difference vector
    $\a$ to accommodate for the immersion and we must take into
    account spherical principal surface.}
\end{figure}
It is possible to augment the results of the calculation from the
previous chapter to treat an aplanatic immersion objective
\citep{Hwang2008}.

We account for the immersion medium by shifting the focal plane in
sample space to $nf$ using the focus difference vector $\a$.
\begin{align}
  \a &= \boxed{f (n-1) \hz} \\
  R &= \boxed{nf}
\end{align}
The principal surface\footnote{An image forming system focusses
  parallel light into a point. Its prinicipal surface is the surface
  where an incident parallel ray intersects with a line along the
  transmitted image forming ray.} is a sphere of radius $R=nf$ around
the image point (\cite{Smith2000} p.~22). In the paper
\citep{Hwang2008} they express the deviation between the real
principal surface and the principal plane with an approximation for
small angles $\theta$ and $\phi$:
\begin{align}
  \s &= \boxed{(R - \sqrt{R^2-\rho^2})\i}
\end{align}
This is an approximation because it only takes into account the
perpendicular (along $\z$) distance between plane and sphere. They
demonstrate the viability of this approximation by comparing its
results with a full raytrace through a $100\times\,1.41$
objective. Focus displacement errors are less than \unit[130]{nm} for
a field of $\unit[86.4]{\mu m}$ radius. This is sufficient for our
problem. As we anyway have the code for a ray--sphere intersection, we
can use it here as well and calculate an exact vector $\s$.

The final ray exiting the objective has the direction $\r_0$:
\begin{align}
  \r_0 &= \boxed{\r + \a - \s}.
\end{align}
\section{Reverse path through oil objective (detection)}
Now we consider the oil objective in the reverse direction (see
\figref{fig:obj-ref-full}). We have a ray starting within the sample
and want to know the transmitted ray in the pupil.

\subsection{Easy case: back focal plane positions only}
If we are only interested in positions of rays in the back focal
plane, we don't have to do full raytracing. If we are imaging beads in
index matched embedding medium and we want to calculate shadow maps
for the MMA (see section \ref{sec:shadow-map}), we don't need a full
raytrace. Instead it is sufficient to ignore ray origins and just
consider their directions.

A unit ray direction $\i=(x,y,z)^T$ in sample space is transformed
into a position $\r_b=(x',y')^T$ in the back focal plane of the
objective. The azimuthal angle $\phi$ isn't changed when going through
the objective. The polar angle $\theta$ defines how far off axis the
back focal plane is hit.
\begin{align}
  \phi'&=\phi=\arctan(y/x)\\
  \theta&=\arcsin(\sqrt{x'^2+y'^2})\\
  r_b&=nf\sin\theta\\
  \r_b&=r_b(\cos\phi',\sin\phi')^T
\end{align}
 \begin{figure}[!hbt]
   \centering
   \input{obj-rev.eps_tex}
   \caption{Schematic for tracing a ray direction $\i$ from sample
     space into the back focal plane. The bigger the angle between
     $\i$ and the optical axis, the further outside the ray will pass
     through the back focal plane.}
 \end{figure}
\subsection{Full raytrace}
If we are also interested in the angles of the transmitted rays in the
back focal plane, when we want to trace the rays further into the
camera or if we want to consider aberrations due to an index mismatch
of the embedding medium, we will have to calculate a full raytrace, as
describted below.

The position of the objective is defined by its principal point $\c$
and the normal $\n$ (directed along optical axis towards sample
space). The incident ray is defined by its starting point $\p$ and the
direction $\i$. First we calculate the center of the gaussian sphere
$\vect g$:
\begin{align}
  \vect g &= \c + nf \n.
\end{align}
Then we obtain the position $\p'$ by intersecting the incident ray and
the plane perpendicular to the optical axis through $\vect{g}$.  The focus
difference vector is defined by its length and the optical axis. It
can be used to calculate an intermediate point $\p''$.
\begin{align}
  \a &= -f(n-1)\n \\
  \p'' &= \p' + a.
\end{align}
The point $\p''$ has now been shifted, so that a thin air lens would
image it exactly as the oil objective would image $\p'$. We can use
$\p''$ to find the direction $\t$ of the transmitted ray. It is just
the normalized difference vector $\vect m$ to the principal point.
\begin{align}
  \vect m &= \c - \p'' \\
  \t &= \vect m / \abs{\vect m}.
\end{align}
As a last step we calculate the starting point $\e$ of the transmitted
ray by intersecting the incident ray with the gaussian sphere.
\begin{figure}[!hbt]
  \centering
  \input{obj-rev-full.eps_tex}
  \caption{Construction to find the transmitted ray through an oil
    immersion objective from a point within the sample.}
  \label{fig:obj-ref-full}
\end{figure}
\subsection{Treatment of aberration (detection)}
Now we consider a ray originating in point $\p$ with direction $\i$
within an immersion of index $n_e$. We want to treat the problem of a
non-matched embedding medium $n_e\not=n$. We find the intersection
$\f$ of the ray with the coverslip--embedding interface and refract to
obtain $\i'$. We calculate the time $t$ a photon takes, to travel from
$\p$ to the interface $\p$:
\begin{align}
  t = \abs{\f - \p} n_e c
\end{align}
and extend the path of the photon backward along $\i'$ by
$t/(cn)$. This results in the corrected position $\p'$ that indicates
where the photon would have originated if the embedding would have
been index matched.  Now we can apply the equations from the previous
sections on the ray defined by $\p'$ and $\i'$ to obtain the
transmitted ray in the pupil.

 \begin{figure}[!hbt]
   \centering
   \input{obj-rev-full-emb.eps_tex}
   \caption{Construction that treats the interface between embedding
     and immersion medium}
 \end{figure}
\section{Sphere projection}
When we model our sample as a collection of spheres, it is useful to
trace rays from the periphery of these spheres through an in focus
target $\c$ into the back focal plane. Here we construct the rays.

The tangents of an out of focus sphere $S^\s_r$ centered at $\s$ with
radius $r$ that pass through the target $\c$ form a double cone
(assuming $\c$ is outside of $S^\s_r$. The tangents touch the surface
of the sphere $S^\s_r$ at the circular intersection $C$ with the sphere
$S^\c_R$ centered at $\c$ with radius $R=\abs{\c-\s}$. Radius $R$ is
the distance from the target to the center of the out of focus sphere.
\begin{figure}[!hbt]
  \centering
  \input{touch-cone.eps_tex}
  \caption{Schematic of how an out of focus nucleus defines a cone of
    tangential rays.}
\end{figure}
In order to find a point $\e$ where a tangent touches the out of focus
sphere, it is sufficient to solve the following equation in a 2D
coordinate system with the origin in the center $\s$ of the out of
focus sphere:
\begin{align}
  (x-R)^2+y^2&=R^2\\
  x^2+y^2=r^2
\end{align}
There are two solutions:
\begin{align}
  x_1&=\frac{r^2}{2R}\label{eqn:x1}\\ 
  y_{1/2}&=\pm\frac{r}{2R}\sqrt{4R^2-r^2} \label{eqn:y1}
\end{align}
In the case $R<r$ the out of focus nucleus is intersecting the target,
obliviating the reason to do the projection in the first place.

We construct two vectors $\hx$ and $\hy$ in order to transform the
solution from 2D into 3D. The (unnormalized) direction $\x$ of the
x-axis of this coordinate system is given by the difference vector of
the target $\c$ and the nucleus center $\s$. The direction $\y$ must
be perpendicular to $\x$ and is obtained by calculating the cross
product with another vector $\q$.  We ensure that $\q$ and $\x$ are
not colinear. The vectors $\q$ and $\x$ are colinear, when the
absolute value of their scalar product equals the square of the length
$\abs{\q\x}=\x^2$.
\begin{align}
  \x&=\c-\s\\
  \q&=\begin{cases}
    (0,0,1)^T & \textrm{when}\ \abs{x_z}<\frac{2}{3}\abs{\x}\\
    (0,1,0)^T & \textrm{else}
  \end{cases}\\
  \y&=\x\times\q \\
  \hx&=\x/\abs{\x}\\
  \hy&=\y/\abs{\y}
\end{align}
Now we can sample the intersection circle $C$ in order to create
viable starting points $\e$ for tangential rays.  Let $R_\phi^\hc$ be
a rotation matrix that rotates a vector by angle $\phi$ around an axis
$\hc$. A point $\e$ on the circle is then defined using one solution
from equations \ref{eqn:x1} and \ref{eqn:y1}. The ray direction $\f$
is then easily obtained:
\begin{align}
  \e&=\s+x_1\hx+y_1R_\phi^\hx\hy\\
  \f&=\c-\e.
\end{align}
Tracing a sufficient number of rays (e.g.\ 16) with direction $\f$ for
different angles $\phi$ to the back focal plane gives the projection
of the intersection circle $C$. Note that this projection in general
is not a circle anymore.

For practical reasons its useful to project vector $\x$ as well. It
can be used as the center of the (distorted) shape on the back focal
plane to rasterize it as a fan of triangles.

%FIXME maybe compare to ./cyberpower-store/0314/zeiss-patents/20080106795-correction-ring.pdf 
%or US7268953-63x.pdf


