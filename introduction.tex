\chapter{Introduction}
Fluorescence microscopy is an old technique that has been established
in live sciences for a long time. Being able to see things happening
at the micrometre scale is the fundamental path to understand life and
disease.

Innovation continuously improves microscopy and occasionally new
fields of research are opened up: The discovery and development of
fluorescent proteins initiated a revolution in how microscopy can be
applied in living specimen. 

Optical high resolution techniques allow
to observe biological processes at the scale of individual molecules
(tens of nm). 

Labels that report membrane potentials or viscosity within cells,
compounds that locally release chemicals when illuminated by light or
even ion pumps that can be switched by light promise novel interesting
research.

All these techniques have in common, that excitation light has to
reach a focal point, line or plane within the sample. For this the
light has to traverse a more or less dense distribution of
fluorophores.

With few exceptions (2-photon, SPIM, CLEM) microscopes are generally
not optimized for exciting only the in focus fluorophores.

Here we build a spatio-angular microscope. It contains two
programmable masks and can control which parts of the sample (spatial)
are illuminated by what angles. The advantages of this microscope lie
in imaging of living samples with sparse, three-dimensionally
extended labelling.