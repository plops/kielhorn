\chapter{Introduction}
Fluorescence microscopy is an old technique that has been established
in live sciences for a long time. Being able to see things happening
at the micrometer scale is the fundamental path to understand life and
disease.

Innovation continuously improves microscopy and occasionally new
fields of research are opened up: The discovery and development of
fluorescent proteins initiated a revolution in how microscopy can be
applied in living specimen. Optical high resolution techniques allow
to observe biological processes at the scale of individual molecules
(tens of nm). 

Labels that report membrane potentials or viscosity within cells,
compounds that locally release (uncage) chemicals when illuminated by
light or even ion pumps that can be switched by light promise novel
interesting research.

All these techniques have in common, that excitation light has to
reach a focal point, line or plane within the sample which consists of
a more or less dense fluorophore distribution.

With few exceptions (2 photon, SPIM, CLEM) microscopes are generally not too
intelligent about what other fluorophores are excited, apart from the
ones in focus.

Here we build a spatio-angular microscope. It contains two
programmable masks and can control which parts of the sample (spatial)
are illuminated by what angles. Its applications will mostly cover
imaging of living organisms experiments in them. There the effects of
phototoxicity are strongest and our microscope has an advantage over
conventional techniques.
